% !TeX root = master.tex

\section{RacePerformance (VMG evaluator)}

This section describes how RacePerformance turns a stream of heading and speed samples into a
simple, stable evaluation: how much the most recent behavior improves or degrades VMG
relative to a low-pass baseline. The goal is not to estimate true wind, but to provide a
consistent reference so changes in trim or steering show up as a clear percentage.

\subsection{Signals and baseline filters}

This subsection defines the signals used by the evaluator and how they are filtered.

RacePerformance applies a first-order low-pass baseline and computes deviations:
\begin{flalign}
\hspace{5em}\alpha_k &= \frac{\Delta t_k}{\tau+\Delta t_k} && \left[1\right] \label{eq:vmg:alpha}
\end{flalign}
In \eqref{eq:vmg:alpha}, \(\alpha_k\) is the per-sample gain, \(\Delta t_k\) is the sample
interval, the subscript \(k\) is the sample index, and \(\tau\) is the user-selected time
constant. This gain is applied to the speed baseline in the next step.
\begin{flalign}
\hspace{5em}\overline{v}_k &= \overline{v}_{k-1}+\alpha_k\left(v_k-\overline{v}_{k-1}\right) && \left[\text{m/s}\right] \label{eq:vmg:vbar-update}
\end{flalign}
In \eqref{eq:vmg:vbar-update}, \(v_k\) is the instantaneous boat speed from the Kalman
state, optionally enhanced by the device motion sensor, and \(\overline{v}_k\) is the
low-pass speed baseline. The same gain is then applied to heading with angular wrap.
\begin{flalign}
\hspace{5em}\overline{h}_k &= \overline{h}_{k-1}+\alpha_k\,\mathrm{wrap}\left(h_k-\overline{h}_{k-1}\right) && \left[\text{rad}\right] \label{eq:vmg:hbar-update}
\end{flalign}
In \eqref{eq:vmg:hbar-update}, \(h_k\) is the instantaneous heading, \(\overline{h}_k\) is
the baseline heading, and \(\mathrm{wrap}(\cdot)\) returns the shortest signed angular
difference. The filters start from the first sample to avoid a delayed plot.
\begin{flalign}
\hspace{5em}\overline{v}_0 &= v_0 && \left[\text{m/s}\right] \label{eq:vmg:vbar-init}
\end{flalign}
In \eqref{eq:vmg:vbar-init}, the subscript \(0\) denotes the first sample used to
initialize the speed baseline. The heading baseline is initialized in the same way.
\begin{flalign}
\hspace{5em}\overline{h}_0 &= h_0 && \left[\text{rad}\right] \label{eq:vmg:hbar-init}
\end{flalign}
In \eqref{eq:vmg:hbar-init}, the first heading sample anchors the baseline. With
baselines defined, we compute the deviations.
\begin{flalign}
\hspace{5em}\Delta v_k &= v_k-\overline{v}_k && \left[\text{m/s}\right] \label{eq:vmg:dv}
\end{flalign}
In \eqref{eq:vmg:dv}, \(\Delta v_k\) is the speed deviation from the baseline. The
heading deviation follows.
\begin{flalign}
\hspace{5em}\Delta h_k &= \mathrm{wrap}\left(h_k-\overline{h}_k\right) && \left[\text{rad}\right] \label{eq:vmg:dh}
\end{flalign}
In \eqref{eq:vmg:dh}, \(\Delta h_k\) is the signed heading deviation, wrapped to the
shortest angular difference. The derivation below omits the sample index for clarity.
The evaluator is driven by the same Kalman output stream as the track view, so it updates
on prediction steps as well as GPS updates.

\subsection{VMG improvement model}
This subsection explains how mode, tack, and the assumed TWA enter the percent-improvement
calculation. It starts by defining the tack-signed target angle, then defines VMG, and
finally derives the improvement expression and its mode-specific forms.

\begin{flalign}
\hspace{5em}a &=
\begin{cases}
-a_{\mathrm{up}}, & \text{beat, port tack} \\
\phantom{-}a_{\mathrm{up}}, & \text{beat, starboard tack} \\
0, & \text{reach mode} \\
-a_{\mathrm{down}}, & \text{run, port tack} \\
\phantom{-}a_{\mathrm{down}}, & \text{run, starboard tack}
\end{cases}
&& \left[\text{rad}\right] \label{eq:vmg:target-angle}
\end{flalign}
In \eqref{eq:vmg:target-angle}, \(a\) is the signed target TWA, \(a_{\mathrm{up}}\) is
the upwind target, and \(a_{\mathrm{down}}\) is the downwind target. Port uses the
negative sign and starboard uses the positive sign. With \(a\) fixed, we define the
baseline VMG.
Reach mode deserves an explicit interpretation. In \eqref{eq:vmg:target-angle} we set
\(a=0\) for reach, which makes the improvement measure track how well you hold the
baseline direction rather than true wind progress. That is a stretch to call VMG: it
assumes you want to keep going where you are going on average, and any deviation from
that baseline is treated as under-performance.
\begin{flalign}
\hspace{5em}\overline{w} &= \overline{v}\cos a && \left[\text{m/s}\right] \label{eq:vmg:wbar}
\end{flalign}
In \eqref{eq:vmg:wbar}, \(\overline{w}\) is the baseline VMG and \(\overline{v}\) is the
low-pass speed baseline. The instantaneous VMG uses the current speed and heading
deviation.
\begin{flalign}
\hspace{5em}w &= v\cos\left(a+\Delta h\right) && \left[\text{m/s}\right] \label{eq:vmg:w}
\end{flalign}
In \eqref{eq:vmg:w}, \(w\) is the instantaneous VMG, \(v\) is the instantaneous speed,
and \(\Delta h\) is the heading deviation. The improvement ratio compares instantaneous
to baseline VMG.
\begin{flalign}
\hspace{5em}R &= \frac{w}{\overline{w}} && \left[1\right] \label{eq:vmg:ratio}
\end{flalign}
In \eqref{eq:vmg:ratio}, \(R\) is the VMG ratio. With \(R\) defined, we convert the ratio
to a percentage before returning it to the UI.
\begin{flalign}
\hspace{5em}\eta &= 100\left(R-1\right) && \left[\%\right] \label{eq:vmg:eta}
\end{flalign}
In \eqref{eq:vmg:eta}, \(\eta\) is the percent improvement shown in the UI. The mode
choice only selects the target angle \(a\) using \eqref{eq:vmg:target-angle}: reach sets
\(a=0\), beat uses the upwind target with a negative sign on port and a positive sign on
starboard, and run uses the downwind target with the same sign rule. In run mode the
target angle exceeds ninety degrees, so the cosine term is negative; both baseline and
instantaneous VMG share that sign, and the ratio still reports a positive improvement for
faster or better-aligned running.

\subsection{Displayed signal and plot window}

This subsection describes the optional smoothing of the displayed signal and how the time
history window is chosen.

The baseline filters always use the long time constant chosen by the user. The displayed
signal can either be the raw improvement or a ``fast'' smoothed version that uses a
shorter time constant.
\begin{flalign}
\hspace{5em}\tau_{\mathrm{fast}} &= \frac{\tau}{10} && \left[\text{s}\right] \label{eq:vmg:tau-fast}
\end{flalign}
In \eqref{eq:vmg:tau-fast}, \(\tau_{\mathrm{fast}}\) is used only for the display; the
baseline filters remain set by \(\tau\). If the baseline magnitude is too small, the
output is suppressed until it stabilizes.

The plot window length is defined by the main time constant.
\begin{flalign}
\hspace{5em}T_{\mathrm{window}} &= 2\tau && \left[\text{s}\right] \label{eq:vmg:window}
\end{flalign}
In \eqref{eq:vmg:window}, \(T_{\mathrm{window}}\) is the fixed plot span of two time
constants. The time axis
runs left to right with ``now'' at the right edge. We draw the improvement history as a
line with a filled area underneath: green for positive values and red for negative
values. The percent axis is symmetric with labeled ticks. An optional cap limits the
displayed magnitude to a fixed percentage so outliers do not rescale the plot.

\subsection{Heading sources and device motion}

This subsection describes how heading is sourced and how the device motion option
interacts with GPS.

RacePerformance always uses the Kalman heading. When the device motion sensor is enabled, we
integrate yaw rate to track
short-term heading changes and blend it back toward GPS heading when speed is sufficient
and accuracy is acceptable. This improves responsiveness when the phone is rigidly
mounted to the boat, but it can degrade estimates if the phone is hand-held or moved
relative to the boat, which is why the UI warns about mounting.
