% !TeX root = master.tex

\section{RaceLifter (headers and lifts)}

This section explains how RaceLifter turns heading samples into a stable lift/shift view.
The aim is to show how much the heading has moved relative to a recent mean, without the
plot jumping around when the window size changes or when a new sample lands.

\subsection{Heading bins and mean}

This subsection describes how samples are binned in time and how the mean heading is
computed.

RaceLifter collects heading samples from the selected source (boat model or GPS) whenever
speed exceeds a small threshold (about \(0.5\,\mathrm{m/s}\)), so drifting or stationary
noise does not dominate the plot. Samples are assigned to time bins based on timestamp,
with the bin duration determined by the current window length. Each bin stores a circular
mean of its samples so headings that straddle \(0^\circ\) do not average incorrectly.

The mean heading used to center the plot is computed from the oldest bins in the current
window, explicitly excluding the active bin at the top. This keeps the reference stable:
new samples only affect the active bin until it is closed, and the plot only recenters
when the window is re-binned.

\subsection{Scaling and bar rendering}

This subsection explains the visual rules that keep the bar plot legible and stable.

Each bin is drawn as a horizontal bar whose length is the signed deviation from the mean.
The scale is symmetric and rounded to fixed step sizes, so the dashed grid aligns with
clean increments and does not jitter as the data shifts. The active bin is clipped to the
current scale so a transient spike does not rescale the entire plot, and only a window
change triggers a full rescale across all bars.

\subsection{Debug playback}

This subsection outlines how the debug mode feeds the same plotting pipeline.

When debug mode is enabled, replay is controlled from the RaceTools page. The replay
button opens a list of recorded sessions pulled from files listed in
\texttt{replay/manifest.json}, and the selected session is streamed back into the app as
time-stamped samples. RaceLifter receives those samples through the same heading pipeline
as live data, so the binning and scaling behavior stays consistent. The replay speed
slider adjusts the playback time base, while the render cadence stays capped to the same
per-second pacing used on the water, keeping the plot stable during tuning.
