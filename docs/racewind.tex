% !TeX root = master.tex

\section{RaceWind wind analysis}

This section explains how RaceWind helps a sailor answer a practical question: is the wind
oscillating in a way that can be timed, or is it too irregular to trust a rhythm? The method is
designed to surface repeatable periods, turn them into a compact reconstruction overlay, and report
how much of the recent variation that overlay explains. Each step below is chosen to support that
goal.

RaceWind uses the same wind history the rest of the view relies on. The history window sets which
samples participate in the analysis, while the period cap limits which candidate periods are
considered for the reconstruction. This separation keeps the model focused on actionable cycles
without discarding input data.

\subsection{Detrending and centering}

This subsection explains how the signal is detrended and mean-centered before any spectral
computation so long-term drift does not dominate the output.

RaceWind removes an offset plus a strictly linear term in time from the samples, then subtracts
the mean of the detrended series. We refer to the full expression as an affine trend, but the
linear part is the through-origin term only. For the periodogram, the model is fit against actual
timestamps.

\begin{flalign}
\hspace{5em}\tilde{x}_i &= x_i - \left(a + b\,t_i\right) && \left[u\right] \label{eq:racewind:detrend}
\end{flalign}
In \eqref{eq:racewind:detrend}, \(x_i\) is the raw series, \(\tilde{x}_i\) is the detrended series,
\(t_i\) is the time coordinate, and \(u\) is the unit of the signal (knots for wind speed or degrees
for unwrapped direction). After detrending, the mean of \(\tilde{x}_i\) is removed so the resulting
series is centered on zero. This centered signal is the starting point for period selection and
model fitting, which are covered next.

\subsection{Lomb--Scargle periodogram}

This subsection explains the Lomb--Scargle periodogram used to detect periodic structure in the
irregularly sampled wind speed data, with an emphasis on why we use it and how it feeds the
reconstruction overlay.

The goal of the periodogram in RaceWind is not to estimate a full spectral density; it is a
diagnostic lens for spotting repeated patterns in wind speed that might matter over the next few
minutes. We prefer a method that tolerates uneven sampling, missing points, and latency spikes,
because those are realities of live wind feeds. A naive FFT requires a uniform grid, and forcing
the data onto that grid either invents samples or drops time stamps, both of which can shift peaks
and blur what the sailor is trying to spot. Lomb--Scargle keeps the raw time stamps, so the period
axis stays honest even if the sampling intervals are not. Importantly, we use the periodogram as a
frequency-selection tool inside a greedy loop: we identify the strongest peak, subtract its fitted
sinusoid from the signal, and then recompute the periodogram on the residual. That strategy favors
distinct structure over redundant neighbors and yields a compact set of candidate periods that can
be refit together afterward.

Another design choice is to show periods rather than frequencies. Sailors reason about “every
5 minutes” or “every 90 seconds,” not “0.2 Hz.” We therefore evaluate the spectrum on a frequency
grid but present its inverse on the axis. The period cap slider lets you ignore long cycles that
are not actionable in the moment, while still using the full history window so short-period peaks
are computed from as much data as possible.

The Lomb--Scargle formulation can be viewed as a least-squares fit of sine and cosine waves at each
test frequency. Although a cosine and sine pair already spans any phase, the phase offset \(\tau\)
is still essential with irregular sampling: it re-centers the basis so the cosine and sine columns
are as orthogonal as possible under the actual time stamps. That improves numerical conditioning
and makes the power comparable across frequencies, rather than letting sampling skew one phase
direction. The offset is computed for each angular frequency \(\omega\):

\begin{flalign}
\hspace{5em}\tau &= \frac{1}{2\omega}\tan^{-1}\left(\frac{\sum_i \sin(2\omega t_i)}{\sum_i \cos(2\omega t_i)}\right) && \left[\mathrm{s}\right] \label{eq:racewind:ls-tau}
\end{flalign}
With \(\tau\) from \eqref{eq:racewind:ls-tau}, we first define the cosine and sine projections and
their normalizers:

\begin{flalign}
\hspace{5em}C(\omega) &= \sum_i \tilde{x}_i\cos\left(\omega(t_i-\tau)\right) && \left[u\right] \label{eq:racewind:ls-cos-proj}
\end{flalign}
\begin{flalign}
\hspace{5em}S(\omega) &= \sum_i \tilde{x}_i\sin\left(\omega(t_i-\tau)\right) && \left[u\right] \label{eq:racewind:ls-sin-proj}
\end{flalign}
\begin{flalign}
\hspace{5em}D_c(\omega) &= \sum_i \cos^2\left(\omega(t_i-\tau)\right) && \left[1\right] \label{eq:racewind:ls-cos-denom}
\end{flalign}
\begin{flalign}
\hspace{5em}D_s(\omega) &= \sum_i \sin^2\left(\omega(t_i-\tau)\right) && \left[1\right] \label{eq:racewind:ls-sin-denom}
\end{flalign}
In \eqref{eq:racewind:ls-cos-proj} and \eqref{eq:racewind:ls-sin-proj}, \(C(\omega)\) and \(S(\omega)\)
are the least-squares projections of the detrended series onto cosine and sine bases. The terms
\(D_c(\omega)\) and \(D_s(\omega)\) in \eqref{eq:racewind:ls-cos-denom} and \eqref{eq:racewind:ls-sin-denom}
are the corresponding energy normalizers that keep each projection scale-consistent even with
irregular sampling.

With those pieces, the Lomb--Scargle power is:

\begin{flalign}
\hspace{5em}P(\omega) &= \frac{1}{2\sigma^2}\left[\frac{C(\omega)^2}{D_c(\omega)} + \frac{S(\omega)^2}{D_s(\omega)}\right] && \left[1\right] \label{eq:racewind:ls-power}
\end{flalign}
In \eqref{eq:racewind:ls-power}, \(P(\omega)\) is the normalized power and \(\sigma^2\) is the
variance of the detrended, centered series. The normalization keeps the scale readable and makes a
peak at one frequency comparable to a peak at another, even if the absolute wind variance changes
through the day.

The periodogram is also intentionally bounded in both directions. Very short periods can be
dominated by noise or sensor jitter, while very long periods can be hard to validate with the
available history. RaceWind therefore sets a minimum period based on the median sampling interval
and caps the maximum period to the slider setting. These bounds are not meant to be theoretical
limits; they are practical guards against over-interpreting either high-frequency noise or
slow drift.

Finally, it is worth stating how to interpret the peaks themselves. A peak indicates a repeating
pattern in the wind speed at the corresponding period, but it does not guarantee that the
oscillation will continue. The analysis is a snapshot of the recent window, not a forecast. It is
most valuable when you see a peak that persists over several updates while the median wind remains
stable, because that suggests a durable cycle worth timing your maneuvers around. Conversely, when
peaks jump around or wash out, the periodogram is telling you the wind is too irregular for a
reliable cycle-based tactic.

At this stage we have an ordered list of candidate periods produced by the greedy residual search.
The next step is to turn those candidates into a single reconstruction that can be compared to the
live wind trace.

\subsection{Reconstructed speed overlay}

This subsection explains the reconstructed wind speed overlay that is drawn on top of the raw
speed history to visualize the dominant periodic components and to quantify how predictable the
wind appears to be.

RaceWind selects candidate periods by the greedy residual method described above, then fits all of
the selected frequencies together in a single least-squares solve. We build a design matrix with
cosine and sine columns for each selected frequency, solve for all coefficients at once, and then
sum the resulting sinusoids. This means the amplitudes and phases are optimized jointly rather
than one frequency at a time. The reconstruction is the sum of those sinusoids, with the original
affine trend and mean added back so the overlay sits in the same amplitude range as the measured
wind speed. This two-step flow keeps the throughline clear: the periodogram loop proposes promising
periods, and the joint fit is the model that explains as much of the centered signal as possible.
The result is a compact, readable hint of the strongest cycles without pretending to be a forecast.
The line is drawn in the same red accent used elsewhere in the view so it remains distinct from the
raw trace.

The number of peaks included in the reconstruction is selectable per plot. A fit order slider
sets \(M \in [1, 5]\), which controls how many candidate periods the greedy search is allowed to
produce before the joint fit. A smaller \(M\) yields a simpler, more stable reconstruction, while a
larger \(M\) captures more structure but risks fitting transient noise. If the residual search does
not find \(M\) meaningful peaks, the control caps at the number that are available. The overlay and
fit score update immediately when the order changes.

The reconstructed signal can be written as:

\begin{flalign}
\hspace{5em}\hat{x}(t) &= \sum_{m=1}^{M}\left(a_m\cos\left(\omega_m(t-\tau_m)\right) + b_m\sin\left(\omega_m(t-\tau_m)\right)\right) + \gamma + \beta t && \left[u\right] \label{eq:racewind:reconstruction}
\end{flalign}
In \eqref{eq:racewind:reconstruction}, \(\omega_m\) are the \(M\) angular frequencies selected by the
greedy residual search, \(a_m\) and \(b_m\) are the jointly fitted cosine and sine coefficients,
\(\tau_m\) are the phase offsets used to align each basis function, \(\gamma\) is the constant
offset that combines the detrended mean with the affine intercept, and \(\beta t\) is the strictly
linear through-origin trend component that was removed before the spectral analysis.
This overlay shares the same history window as the wind plots, so it updates whenever new samples
arrive or the horizon is adjusted via the zoom controls.

Once the reconstruction is assembled, RaceWind also reports a goodness-of-fit value. The fit score
is computed on the detrended, mean-centered series, so it reflects how much of the remaining
variation is explained by the \(M\)-period model rather than by slow drift or offset.

To make the fitting step explicit, we assemble a design matrix \(A\) from the cosine and sine basis
terms evaluated at each sample time:

\begin{flalign}
\hspace{5em}A_{i,2m-1} &= \cos\left(\omega_m(t_i-\tau_m)\right) && \left[1\right] \label{eq:racewind:fit-matrix-cos}
\end{flalign}
\begin{flalign}
\hspace{5em}A_{i,2m} &= \sin\left(\omega_m(t_i-\tau_m)\right) && \left[1\right] \label{eq:racewind:fit-matrix-sin}
\end{flalign}
In \eqref{eq:racewind:fit-matrix-cos} and \eqref{eq:racewind:fit-matrix-sin}, \(i\) indexes samples in
the centered series, and \(m \in \{1,\dots,M\}\) indexes the selected frequencies. With the centered
data vector \(y\) (detrended and mean-removed), the coefficient vector
\(\beta = [a_1, b_1, \dots, a_M, b_M]^\mathsf{T}\) is obtained by the Moore--Penrose least-squares
solution:

\begin{flalign}
\hspace{5em}\beta &= \left(A^\mathsf{T}A\right)^{-1}A^\mathsf{T}y && \left[u\right] \label{eq:racewind:fit-solve}
\end{flalign}
In \eqref{eq:racewind:fit-solve}, \(u\) is the unit of the centered signal, so each coefficient is
expressed in knots for speed or degrees for direction. This explicit construction is the joint fit
step referenced above. The same reconstruction procedure is applied to the unwrapped wind direction
series, with its own fit order control and the same interpretation for periods and amplitudes.
