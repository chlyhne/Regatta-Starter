
\hypertarget{kalman-filter-math-physical-arguments-and-gain-scheduling}{%
\section{Kalman Filter: Math, Physical Arguments, and Gain
Scheduling}\label{kalman-filter-math-physical-arguments-and-gain-scheduling}}

RaceTimer uses a small Kalman filter to turn ``jittery GPS fixes'' into
a \textbf{steady position} and a \textbf{stable velocity vector}. The
aim is not academic perfection; it is a practical tool for sailors:

\begin{itemize}
\tightlist
\item
  numbers should not jump around when you need to make decisions quickly
\item
  direction and speed estimates should not flip wildly from one fix to
  the next
\item
  the filter should adapt automatically when GPS quality changes
\end{itemize}

This document is written as an article. It starts from first principles,
then ties each assumption directly to the code. Tuning details live in
\texttt{docs/tuning.tex}.

If you only want the brief version, see the ``Kalman filter'' section in
\texttt{README.md}.

\hypertarget{why-a-kalman-filter-at-all}{%
\subsection{Why a Kalman filter at
all?}\label{why-a-kalman-filter-at-all}}

GPS position updates are noisy. If you derive speed and heading from
consecutive fixes, you amplify that noise. Near a start line, small
errors become big problems:

\begin{itemize}
\tightlist
\item
  distance-to-line can jitter by multiple meters
\item
  heading can wobble, making ``current heading'' projections unstable
\item
  the app becomes stressful to use because the numbers are not
  ``trustworthy''
\end{itemize}

The Kalman filter is a compact way to combine:

\begin{enumerate}
\def\labelenumi{\arabic{enumi})}
\tightlist
\item
  a simple motion model (``boats don't teleport; they move smoothly''),
  and\\
\item
  measurements (``GPS says we are here, but with some uncertainty'').
\end{enumerate}

The filter's job is to produce the best compromise \textbf{for the
use-case}.

\hypertarget{coordinate-system-turning-latitudelongitude-into-meters}{%
\subsection{Coordinate system: turning latitude/longitude into
meters}\label{coordinate-system-turning-latitudelongitude-into-meters}}

The filter runs in a local tangent-plane approximation:

\begin{itemize}
\tightlist
\item
  choose a local origin near the current area
\item
  convert \((\text{lat}, \text{lon})\) into \((x, y)\) meters relative
  to that origin
\end{itemize}

This is implemented with helpers in \texttt{geo.js} (\texttt{toMeters},
\texttt{fromMeters}). Over the distances we care about at a start line,
this is accurate enough and avoids complicated geodesy inside the
filter.

\hypertarget{what-the-filter-estimates-the-state}{%
\subsection{1) What the filter estimates (the
state)}\label{what-the-filter-estimates-the-state}}

We estimate both position and velocity in 2D:

\[
x = [p_x, p_y, v_x, v_y]^T
\]

Units: - \(p_x, p_y\) in meters - \(v_x, v_y\) in meters/second

This means the filter can produce a steady velocity vector even when GPS
does not provide a reliable heading/speed (e.g., very low speed, or a
device/browser that omits those fields).

\hypertarget{the-motion-model-constant-velocity}{%
\subsection{2) The motion model (constant
velocity)}\label{the-motion-model-constant-velocity}}

Between GPS updates, we assume the boat maintains constant velocity over
a short interval \(dt\). That is:

\[
\begin{aligned}
p_{x,k} &= p_{x,k-1} + v_{x,k-1}\,dt \\
p_{y,k} &= p_{y,k-1} + v_{y,k-1}\,dt \\
v_{x,k} &= v_{x,k-1} \\
v_{y,k} &= v_{y,k-1}
\end{aligned}
\]

In matrix form:

\[
x_k = F x_{k-1} + w
\]

with

\[
F =
\begin{bmatrix}
1 & 0 & dt & 0 \\
0 & 1 & 0 & dt \\
0 & 0 & 1 & 0 \\
0 & 0 & 0 & 1
\end{bmatrix}
\]

The term \texttt{w} is ``everything the model does not capture'':
acceleration, turning, waves, gusts, steering corrections, and also
model mismatch.

That is what \textbf{process noise} is for.

\hypertarget{process-noise-q-modelling-unknown-acceleration}{%
\subsection{3) Process noise Q: modelling unknown
acceleration}\label{process-noise-q-modelling-unknown-acceleration}}

\hypertarget{the-interpretation-of-q}{%
\subsubsection{\texorpdfstring{The interpretation of
\(q\)}{The interpretation of q}}\label{the-interpretation-of-q}}

We use a standard ``nearly constant velocity'' (CV) model where
acceleration is modeled as continuous white noise. In this model:

\begin{itemize}
\tightlist
\item
  \(q\) is the \textbf{acceleration variance} (units
  \((\mathrm{m/s^2})^2\))
\item
  larger \(q\) means we expect larger unmodeled accelerations
\item
  larger \(q\) makes the filter more willing to change the velocity
  estimate quickly
\end{itemize}

You can think of it as: ``how twitchy is the boat allowed to be?''

\hypertarget{deriving-the-discrete-time-q-matrix}{%
\subsubsection{Deriving the discrete-time Q
matrix}\label{deriving-the-discrete-time-q-matrix}}

For one axis (say \(x\)), with state \([p, v]^T\), constant velocity
model, and white acceleration noise, the discrete-time process noise
covariance becomes:

\[
Q_{\text{axis}} = q
\begin{bmatrix}
dt^4/4 & dt^3/2 \\
dt^3/2 & dt^2
\end{bmatrix}
\]

Why those powers of \(dt\)? - position is the integral of velocity -
velocity is the integral of acceleration - integrating white noise
introduces these time scalings

If we assumed identical behavior in every direction, the full 2D filter
would be two independent copies (\(x\) and \(y\)), combined into
\(4\times4\) form: \[
Q =
\begin{bmatrix}
q\,dt^4/4 & 0 & q\,dt^3/2 & 0 \\
0 & q\,dt^4/4 & 0 & q\,dt^3/2 \\
q\,dt^3/2 & 0 & q\,dt^2 & 0 \\
0 & q\,dt^3/2 & 0 & q\,dt^2
\end{bmatrix}
\]

In RaceTimer we make a more boat-like assumption: \textbf{forward and
sideways acceleration are not equally likely}. We therefore:

\begin{itemize}
\tightlist
\item
  set a forward acceleration variance \(q_f\)
\item
  set a lateral acceleration variance \(q_l\) (typically smaller)
\item
  rotate this anisotropic covariance into the global x/y frame using the
  current heading
\end{itemize}

The matrix is still symmetric and uses the same CV structure, but it now
contains off-diagonal terms because the ``forward'' axis is not aligned
with the global x/y axes.

\hypertarget{measurement-model-gps-gives-position-with-accuracy}{%
\subsection{4) Measurement model: GPS gives position (with
accuracy)}\label{measurement-model-gps-gives-position-with-accuracy}}

GPS provides a measurement of position:

\[
z = [p_x, p_y]^T
\]

and we use:

\[
z_k = H x_k + v
\]

with

\[
H =
\begin{bmatrix}
1 & 0 & 0 & 0 \\
0 & 1 & 0 & 0
\end{bmatrix}
\]

The measurement noise \texttt{v} has covariance \texttt{R}.

\hypertarget{how-r-is-chosen-from-gps-accuracy}{%
\subsubsection{How R is chosen from GPS
accuracy}\label{how-r-is-chosen-from-gps-accuracy}}

Browsers provide \texttt{position.coords.accuracy} in meters. It is
(roughly) a \(1\sigma\) radius. We make a simple, explicit assumption:

\begin{itemize}
\tightlist
\item
  uncertainty is isotropic (same in x and y)
\item
  uncertainty is uncorrelated between x and y
\end{itemize}

So:

\[
R = r I_2, \quad r = \text{accuracy}^2
\]

We clamp the reported accuracy because sometimes devices report absurd
values:

\[
\text{accuracy} = \text{clamp}(\text{reportedAccuracy}, \text{min}, \text{max})
\]

This prevents one bad fix from causing a huge gain swing.

What this means for the Kalman gain: - good GPS (small accuracy)
\(\Rightarrow\) small \(R\) \(\Rightarrow\) higher gain \(\Rightarrow\)
filter follows measurements more - bad GPS (large accuracy)
\(\Rightarrow\) large \(R\) \(\Rightarrow\) lower gain \(\Rightarrow\)
filter trusts the model more

\hypertarget{the-kalman-filter-equations-what-happens-each-gps-update}{%
\subsection{5) The Kalman filter equations (what happens each GPS
update)}\label{the-kalman-filter-equations-what-happens-each-gps-update}}

The filter keeps: - state estimate \texttt{x} - covariance estimate
\texttt{P}

Each new GPS fix performs:

\hypertarget{predict}{%
\subsubsection{Predict}\label{predict}}

\[
x^- = F x
\]

\[
P^- = F P F^T + Q
\]

\hypertarget{update}{%
\subsubsection{Update}\label{update}}

Innovation covariance:

\[
S = H P^- H^T + R
\]

Gain:

\[
K = P^- H^T S^{-1}
\]

Innovation:

\[
y = z - H x^-
\]

State update:

\[
x = x^- + K y
\]

Covariance update:

\[
P = P^- - K (H P^-)
\]

The code implements these explicitly (no external matrix library)
because: - the matrices are tiny (\(4\times4\), \(2\times2\)) - we want
the app to remain a static PWA with minimal dependencies

\hypertarget{predict-only-updates-at-5-hz-between-fixes}{%
\subsection{6) Predict-only updates at 5 Hz (between
fixes)}\label{predict-only-updates-at-5-hz-between-fixes}}

GPS does not arrive at a fixed rate. The app still needs a stable,
smooth estimate for race view, debug view, and GPS marking. To get that,
we run the \textbf{predict step} on a fixed timer (about 5 Hz) whenever
a filter state exists.

What that means in practice:

\begin{itemize}
\tightlist
\item
  Every \(\approx 200\,\mathrm{ms}\) we advance the state with the
  motion model using \textbf{elapsed time}.
\item
  There is \textbf{no measurement update} in these ticks, so \(R\) is
  unchanged.
\item
  When a real GPS fix arrives, we run a normal measurement update at the
  current time. If the fix timestamp is older than the current filter
  time, we treat it as arriving ``now'' to keep the filter time
  monotonic.
\item
  Large time gaps are capped (the max dt clamp still applies) so a
  single pause does not explode the covariance.
\end{itemize}

This gives the same filter behavior as a classic predict-update loop,
but with a smoother output cadence that does not depend on GPS jitter.

Relevant code: - \texttt{predictKalmanState()} in \texttt{kalman.js}
(predict-only step) - the 5 Hz loop in \texttt{app.js} that keeps the
estimate moving

\hypertarget{imu-assisted-heading-optional-racedebug-toggle}{%
\subsection{7) IMU-assisted heading (optional, race/debug
toggle)}\label{imu-assisted-heading-optional-racedebug-toggle}}

The GPS-derived heading can be slow or unstable, especially at low
speed. When IMU assist is enabled, we use the gyroscope's yaw rate to
update the heading estimate between GPS fixes, while still letting GPS
gently correct long-term drift.

\hypertarget{estimating-the-down-axis-continuously}{%
\subsubsection{7.1 Estimating the down axis
continuously}\label{estimating-the-down-axis-continuously}}

The gyroscope reports rotation rates about the device's axes, but we
need the component of rotation about the vertical (down) axis to get
yaw. Because the device can move in waves, we estimate down on every
motion event:

\begin{itemize}
\tightlist
\item
  read \texttt{accelerationIncludingGravity}
\item
  if \texttt{acceleration} is available, subtract it to remove fast
  linear motion\\
  (\(g \approx a_{\text{incl}} - a_{\text{lin}}\))
\item
  low-pass the result so down changes smoothly
\end{itemize}

The low-pass factor is configured in \texttt{docs/tuning.tex} along with
the other tuning constants. This section focuses on how the estimate is
used once tuned.

\hypertarget{converting-gyro-into-yaw-rate}{%
\subsubsection{7.2 Converting gyro into yaw
rate}\label{converting-gyro-into-yaw-rate}}

The device supplies rotation rates around its own axes. We project that
rotation vector onto the estimated gravity direction:

\[
\text{yawRate} \approx - (\omega \cdot \hat{g})
\]

This gives an estimated rotation rate about down. The sign is chosen so
a right turn increases heading in the race/debug view.

\hypertarget{axis-mapping-sanity-check}{%
\paragraph{Axis mapping sanity check}\label{axis-mapping-sanity-check}}

On iOS (screen up, flat on a table) the observed mapping is:

\begin{itemize}
\tightlist
\item
  \(\alpha\) responds to pitch (lift top edge up/down)
\item
  \(\beta\) responds to roll (lift left edge up/down)
\item
  \(\gamma\) responds to yaw (rotate flat on the table)
\end{itemize}

We map these to the internal rotation vector as:

\begin{itemize}
\tightlist
\item
  \(\omega_x = \alpha\)
\item
  \(\omega_y = \beta\)
\item
  \(\omega_z = \gamma\)
\end{itemize}

The gravity projection then extracts the yaw component robustly even
when the device is not perfectly level.

To confirm this on-device, the Debug panel shows:

\begin{itemize}
\tightlist
\item
  \texttt{IMU\ rot}: alpha/beta/gamma in deg/s
\item
  \texttt{IMU\ yaw}: computed yaw rate and the current gravity vector
\end{itemize}

These readouts make it clear if pitch/roll are leaking into yaw and
whether axis mapping needs to be revisited.

\hypertarget{updating-the-filter-with-imu-yaw}{%
\subsubsection{7.3 Updating the filter with IMU
yaw}\label{updating-the-filter-with-imu-yaw}}

When we have a yaw rate:

\begin{itemize}
\tightlist
\item
  update a separate \(\text{headingRad}\)
\item
  rotate the velocity vector by the yaw change
\item
  rotate the velocity covariance block so \texttt{P} stays consistent
\end{itemize}

This is effectively a \textbf{coordinated-turn assumption}: heading
changes redirect the velocity vector while preserving its magnitude and
without introducing sideways speed. For sailboats this is a reasonable
first-order model (the keel resists sideslip), even if reality is not
perfectly lossless.

Once we choose that deterministic rotation, the covariance \textbf{must}
rotate with it. \(P\) is the covariance of the current state vector; if
the state is rotated by a matrix \(A\), the correct covariance is
\(P' = A P A^T\). Not rotating \(P\) would leave the filter internally
inconsistent (uncertainty still aligned to the old velocity direction).

There is also a real-world constraint: the device is fixed to the boat,
so the IMU is measuring \textbf{actual hull rotation}, and the boat's
own inertia acts as a physical low-pass. That helps keep the yaw updates
grounded. We still avoid unphysical instantaneous spins by clamping IMU
dt and blending GPS heading over time.

This does not add a new measurement update; it is a deterministic update
between GPS fixes. The measurement noise \texttt{R} is unchanged.

\hypertarget{blending-gps-heading-and-imu-heading}{%
\subsubsection{7.4 Blending GPS heading and IMU
heading}\label{blending-gps-heading-and-imu-heading}}

When GPS speed is above a minimum threshold, we compute a GPS heading
from the Kalman velocity. If IMU is enabled, GPS nudges the IMU heading
with a tunable weight. If IMU is disabled, GPS heading is used directly.
Tuning details live in \texttt{docs/tuning.tex}.

\hypertarget{calibration-workflow-per-device}{%
\subsubsection{7.5 Calibration workflow (per
device)}\label{calibration-workflow-per-device}}

Different devices report gyro axes differently. To avoid hard-coded
per-device mappings, RaceTimer uses a simple on-device calibration:

\begin{itemize}
\tightlist
\item
  open \textbf{Settings -> IMU calibration}
\item
  place the device flat, screen up
\item
  rotate clockwise for a few seconds
\end{itemize}

The app selects the axis mapping that best aligns the rotation vector
with gravity during that yaw motion and stores it in settings. IMU
assist is blocked until calibration is done.

Calibration also checks for real motion: it requires enough rotation
samples and a consistently positive yaw rate (clockwise). If you do not
rotate the device, calibration fails with an error and the IMU remains
disabled.

\hypertarget{marking-the-start-line-with-gps-no-bow-offset}{%
\subsection{8) Marking the start line with GPS (no bow
offset)}\label{marking-the-start-line-with-gps-no-bow-offset}}

When you press ``Set port mark (GPS)'' or ``Set starboard mark (GPS)'',
the app stores the \textbf{latest Kalman position estimate for the
device}. We deliberately do \textbf{not} apply the bow offset here:

\begin{itemize}
\tightlist
\item
  the user lines up the \textbf{device} with the physical mark
\item
  the most honest reference is therefore the device position itself
\item
  the mark is captured immediately from the current estimate (no
  averaging, no future fixes)
\end{itemize}

Because the Kalman estimate updates at 5 Hz between fixes, the
``latest'' position is a smooth, up-to-date estimate even if GPS
delivers at a slower rate.

\hypertarget{bow-offset-how-it-is-applied-in-race-projections}{%
\subsection{9) Bow offset: how it is applied in race
projections}\label{bow-offset-how-it-is-applied-in-race-projections}}

The Kalman filter estimates the device position and velocity. The boat
bow is then constructed by shifting the device position \textbf{forward
along the velocity vector} by the user's bow offset.

Race view uses two projections, and the bow offset is handled
differently in each:

\begin{enumerate}
\def\labelenumi{\arabic{enumi})}
\item
  \textbf{At current heading}\\
  We already have the bow position (device + forward offset). That bow
  point is projected along the \textbf{current velocity vector} to the
  start time.
\item
  \textbf{Towards line (direct to the line)}\\
  Here we assume you will steer straight toward the closest point on the
  line. We therefore:

  \begin{itemize}
  \tightlist
  \item
    back out the \textbf{device position} from the bow
  \item
    compute the perpendicular direction to the line
  \item
    re-apply the bow offset \textbf{along that direction}, not along the
    current velocity
  \end{itemize}
\end{enumerate}

This keeps the geometry consistent for each assumption.

\hypertarget{why-q-is-rotated-forward-vs-sideways-acceleration}{%
\subsection{\texorpdfstring{10) Why \texttt{Q} is rotated: forward vs
sideways
acceleration}{10) Why Q is rotated: forward vs sideways acceleration}}\label{why-q-is-rotated-forward-vs-sideways-acceleration}}

The \textbf{important modeling choice} is that a boat can change speed
\textbf{much more easily along its heading} than sideways. In other
words, acceleration uncertainty is \textbf{anisotropic}:

\begin{itemize}
\tightlist
\item
  \textbf{Forward (along the boat's heading):} higher variance, because
  speed changes here are common.
\item
  \textbf{Sideways (across the boat):} lower variance, because real
  boats do not slide sideways nearly as much.
\end{itemize}

We encode that by building an anisotropic \texttt{Q} with a forward
variance and a smaller sideways variance, then \textbf{rotating that
covariance into the global x/y frame} using the current velocity
heading. This keeps the model aligned with the boat's direction of
travel and makes the filter less willing to ``invent'' sideways speed.

In the debug view we draw the \textbf{position block of \texttt{Q}} so
this anisotropy is obvious. The overlay is:

\begin{itemize}
\tightlist
\item
  anchored to the \textbf{device position} (the Kalman state is the
  device)
\item
  rotated by the \textbf{current velocity heading} (changes with GPS or
  IMU updates)
\item
  scaled to a fixed display length (currently 10 m) to show orientation
  only
\end{itemize}

We still maintain the full covariance \texttt{P} internally, and it is
generally \textbf{not diagonal} even if \texttt{R} is isotropic (the CV
model couples position/velocity, and updates add correlations). We avoid
plotting \texttt{P} here because its shape is dominated by measurement
updates rather than the model structure we want to inspect.

\hypertarget{where-to-look-in-the-repo}{%
\subsection{11) Where to look in the
repo}\label{where-to-look-in-the-repo}}

\begin{itemize}
\tightlist
\item
  Core filter: \texttt{kalman.js}
\item
  Tuning reference: \texttt{docs/tuning.tex} (values live in
  \texttt{tuning.js})
\item
  Speed history used for scheduling: \texttt{state.speedHistory}
  maintained in \texttt{app.js}
\item
  Debug process-noise visualization: \texttt{track.js}
\end{itemize}

