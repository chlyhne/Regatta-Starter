% !TeX root = master.tex

\section{RacePerformance (VMG evaluator)}

This section describes how RacePerformance turns a stream of heading and speed samples into a
simple, stable evaluation: how much the most recent behavior improves or degrades VMG
relative to a low-pass baseline. The goal is not to estimate true wind, but to provide a
consistent reference so changes in trim or steering show up as a clear percentage.

\subsection{Signals and baseline filters}

This subsection defines the signals used by the evaluator and how they are filtered.

RacePerformance works directly with instantaneous estimates from the navigation layer. Let
\(v\) be the instantaneous boat speed and \(h\) be the instantaneous heading, both sourced
from the same heading selection used elsewhere (boat model or raw GPS). We maintain two
first-order low-pass baselines using the user-selected time constant \(\tau\):
\(\overline{v}=\mathrm{LPF}(v)\) and \(\overline{h}=\mathrm{LPF}(h)\). Both filters are
initialized to the first sample so the plot starts immediately. The instantaneous
deviations are \(dv=v-\overline{v}\) and \(dh=h-\overline{h}\), with \(dh\) wrapped to the
shortest signed angular difference. The evaluator is driven by the same Kalman output
stream as the track view, so it updates on prediction steps as well as GPS updates.

\subsection{VMG improvement model}

This subsection explains how mode, tack, and the assumed TWA enter the percent-improvement
calculation.

RacePerformance does not estimate wind direction directly. Instead, it uses the selected
mode and tack to define a signed target TWA \(a\). In beat and run modes, \(a\) is taken
from the upwind or downwind TWA sliders and signed by tack (port is negative, starboard is
positive). In reach mode we set \(a=0\), which effectively evaluates changes along the
current baseline heading without introducing a wind assumption. The baseline VMG is
\(w=\overline{v}\cos(a)\), and the instantaneous VMG is \(v\cos(a+dh)\).

The ratio of instantaneous VMG to the baseline simplifies to:
\[
\frac{v\cos(a+dh)}{\overline{v}\cos(a)}=
\left(1+\frac{dv}{\overline{v}}\right)\left(\cos(dh)-\tan(a)\sin(dh)\right)
\]
We report percent improvement, so the displayed value is:
\[
X_\%=\left[\left(1+\frac{dv}{\overline{v}}\right)\left(\cos(dh)-\tan(a)\sin(dh)\right)-1\right]\times 100
\]
This single expression handles beat, reach, and run. For running, the cosine in the
denominator is negative, but the ratio removes that sign, so a faster or better-aligned
run produces a positive improvement.

\subsection{Displayed signal and plot window}

This subsection describes the optional smoothing of the displayed signal and how the time
history window is chosen.

The baseline filters always use the long time constant \(\tau\). The displayed signal can
either be the raw improvement \(X_\%\) or a ``fast'' smoothed version using a first-order
low-pass filter with time constant \(\tau/10\). This toggle changes only the display; the
baseline filters are unchanged. If the baseline magnitude is too small, the output is
suppressed until it stabilizes.

The plot window is fixed at four time constants (\(4\tau\)). The time axis runs left to
right with ``now'' at the right edge. We draw the improvement history as a line with a
filled area underneath: green for positive values and red for negative values. The
percent axis is symmetric with labeled ticks. An optional cap limits the displayed
magnitude to a fixed percentage so outliers do not rescale the plot.

\subsection{Heading sources and device motion}

This subsection describes how heading is sourced and how the device motion option
interacts with GPS.

RacePerformance can use the boat model (Kalman) heading or raw GPS heading, depending on the user
selection. When the device motion sensor is enabled, we integrate yaw rate to track
short-term heading changes and blend it back toward GPS heading when speed is sufficient
and accuracy is acceptable. This improves responsiveness when the phone is rigidly
mounted to the boat, but it can degrade estimates if the phone is hand-held or moved
relative to the boat, which is why the UI warns about mounting.
