% !TeX root = master.tex

\section{RaceKBL wind analysis}

This section describes how RaceKBL converts wind history into autocorrelation, cross-correlation,
and period estimates that are legible during a start sequence.

RaceKBL uses the same wind history the rest of the view relies on. The history slider determines
how much data is fed into the analysis, while the autocorrelation and periodogram sliders only cap
the maximum lag or period shown on the plots. This separation lets the plots zoom into a shorter
window without throwing away samples.

\subsection{Detrending and centering}

This subsection explains how the signal is detrended and mean-centered before any correlation or
spectral computation so long-term drift does not dominate the output.

RaceKBL removes a linear trend from the samples, then subtracts the mean of the detrended series.
For uniformly resampled autocorrelation and cross-correlation, the trend is fit against the sample
index (equivalent to uniform time). For the periodogram, the same model is fit against actual
timestamps.

\begin{flalign}
\hspace{5em}\tilde{x}_i &= x_i - \left(a + b\,t_i\right) && \left[u\right] \label{eq:racekbl:detrend}
\end{flalign}
In \eqref{eq:racekbl:detrend}, \(x_i\) is the raw series, \(\tilde{x}_i\) is the detrended series,
\(t_i\) is the time coordinate, and \(u\) is the unit of the signal (knots for wind speed or degrees
for unwrapped direction). After detrending, the mean of \(\tilde{x}_i\) is removed so the resulting
series is centered on zero.

\subsection{Autocorrelation and cross-correlation}

This subsection describes how the autocorrelation and cross-correlation plots are computed and how
their lag axes are capped.

Autocorrelation is normalized by the total variance of the detrended, centered series so the scale
is stable:

\begin{flalign}
\hspace{5em}r_k &= \frac{\sum_{i=0}^{N-k-1}\tilde{x}_i\,\tilde{x}_{i+k}}{\sum_{i=0}^{N-1}\tilde{x}_i^2} && \left[1\right] \label{eq:racekbl:autocorr}
\end{flalign}
In \eqref{eq:racekbl:autocorr}, \(r_k\) is the autocorrelation at lag \(k\), and \(N\) is the number
of samples in the history window. Cross-correlation uses the same detrended data but averages the
product without variance normalization so its magnitude remains in signal units:

\begin{flalign}
\hspace{5em}c_k &= \frac{1}{N_k}\sum_{i=0}^{N_k-1}\tilde{x}_i\,\tilde{y}_{i+k} && \left[u_x\,u_y\right] \label{eq:racekbl:crosscorr}
\end{flalign}
In \eqref{eq:racekbl:crosscorr}, \(c_k\) is the cross-correlation between series \(x\) and \(y\), and
\(N_k\) is the number of overlapping pairs at lag \(k\). Wind direction is unwrapped before this
step so the computation stays smooth across the \(359^\circ/0^\circ\) boundary. The autocorrelation
and cross-correlation sliders cap the maximum lag shown, and the tick spacing is recalculated each
time so the labels match the displayed domain.

\subsection{Lomb--Scargle periodogram}

This subsection explains the Lomb--Scargle periodogram used to detect periodic structure in the
irregularly sampled wind speed data, with an emphasis on why we use it and how it feeds the
reconstruction overlay.

The goal of the periodogram in RaceKBL is not to estimate a full spectral density; it is a
diagnostic lens for spotting repeated patterns in wind speed that might matter over the next few
minutes. We prefer a method that tolerates uneven sampling, missing points, and latency spikes,
because those are realities of live wind feeds. A naive FFT requires a uniform grid, and forcing
the data onto that grid either invents samples or drops time stamps, both of which can shift peaks
and blur what the sailor is trying to spot. Lomb--Scargle keeps the raw time stamps, so the period
axis stays honest even if the sampling intervals are not. Importantly, we use the periodogram as a
frequency-selection step: it tells us which periods are worth modeling, and the reconstruction
pipeline refines the amplitudes and phases afterward.

Another design choice is to show periods rather than frequencies. Sailors reason about “every
5 minutes” or “every 90 seconds,” not “0.2 Hz.” We therefore evaluate the spectrum on a frequency
grid but present its inverse on the axis. The period cap slider lets you ignore long cycles that
are not actionable in the moment, while still using the full history window so short-period peaks
are computed from as much data as possible.

The Lomb--Scargle formulation can be viewed as a least-squares fit of sine and cosine waves at each
test frequency. The phase offset \(\tau\) aligns the basis functions to the data so that the sine
and cosine terms remain orthogonal even with irregular sampling. This makes the power comparable
across frequencies and avoids bias toward specific phases. The offset is computed for each angular
frequency \(\omega\):

\begin{flalign}
\hspace{5em}\tau &= \frac{1}{2\omega}\tan^{-1}\left(\frac{\sum_i \sin(2\omega t_i)}{\sum_i \cos(2\omega t_i)}\right) && \left[\mathrm{s}\right] \label{eq:racekbl:ls-tau}
\end{flalign}
With \(\tau\) from \eqref{eq:racekbl:ls-tau}, we first define the cosine and sine projections and
their normalizers:

\begin{flalign}
\hspace{5em}C(\omega) &= \sum_i \tilde{x}_i\cos\left(\omega(t_i-\tau)\right) && \left[u\right] \label{eq:racekbl:ls-cos-proj}
\end{flalign}
\begin{flalign}
\hspace{5em}S(\omega) &= \sum_i \tilde{x}_i\sin\left(\omega(t_i-\tau)\right) && \left[u\right] \label{eq:racekbl:ls-sin-proj}
\end{flalign}
\begin{flalign}
\hspace{5em}D_c(\omega) &= \sum_i \cos^2\left(\omega(t_i-\tau)\right) && \left[1\right] \label{eq:racekbl:ls-cos-denom}
\end{flalign}
\begin{flalign}
\hspace{5em}D_s(\omega) &= \sum_i \sin^2\left(\omega(t_i-\tau)\right) && \left[1\right] \label{eq:racekbl:ls-sin-denom}
\end{flalign}
In \eqref{eq:racekbl:ls-cos-proj} and \eqref{eq:racekbl:ls-sin-proj}, \(C(\omega)\) and \(S(\omega)\)
are the least-squares projections of the detrended series onto cosine and sine bases. The terms
\(D_c(\omega)\) and \(D_s(\omega)\) in \eqref{eq:racekbl:ls-cos-denom} and \eqref{eq:racekbl:ls-sin-denom}
are the corresponding energy normalizers that keep each projection scale-consistent even with
irregular sampling.

With those pieces, the Lomb--Scargle power is:

\begin{flalign}
\hspace{5em}P(\omega) &= \frac{1}{2\sigma^2}\left[\frac{C(\omega)^2}{D_c(\omega)} + \frac{S(\omega)^2}{D_s(\omega)}\right] && \left[1\right] \label{eq:racekbl:ls-power}
\end{flalign}
In \eqref{eq:racekbl:ls-power}, \(P(\omega)\) is the normalized power and \(\sigma^2\) is the
variance of the detrended, centered series. The normalization keeps the scale readable and makes a
peak at one frequency comparable to a peak at another, even if the absolute wind variance changes
through the day.

The periodogram is also intentionally bounded in both directions. Very short periods can be
dominated by noise or sensor jitter, while very long periods can be hard to validate with the
available history. RaceKBL therefore sets a minimum period based on the median sampling interval
and caps the maximum period to the slider setting. These bounds are not meant to be theoretical
limits; they are practical guards against over-interpreting either high-frequency noise or
slow drift.

Finally, it is worth stating how to interpret the plot. A peak indicates a repeating pattern in
the wind speed at the corresponding period, but it does not guarantee that the oscillation will
continue. The plot is a snapshot of the recent window, not a forecast. It is most valuable when
you see a peak that persists over several updates while the median wind remains stable, because
that suggests a durable cycle worth timing your maneuvers around. Conversely, when peaks jump
around or wash out, the periodogram is telling you the wind is too irregular for a reliable
cycle-based tactic.

\subsection{Reconstructed speed overlay}

This subsection explains the reconstructed wind speed overlay that is drawn on top of the raw
speed history to visualize the dominant periodic components and to quantify how predictable the
wind appears to be.

RaceKBL takes the three most significant Lomb--Scargle peaks and treats them as candidate periods.
Those candidates are then fitted together in a single least-squares solve, so the amplitudes and
phases are optimized jointly rather than one frequency at a time. The reconstruction is the sum of
those three sinusoids, with the original linear trend and mean added back so the overlay sits in the
same amplitude range as the measured wind speed. This two-step flow keeps the throughline clear:
the periodogram is the scout that picks promising periods, and the joint fit is the model that
explains as much of the centered signal as possible. The result is a compact, readable hint of the
strongest cycles without pretending to be a forecast. The line is drawn in magenta so it is visually
distinct from the raw speed trace.

The reconstructed signal can be written as:

\begin{flalign}
\hspace{5em}\hat{x}(t) &= \sum_{m=1}^{3}\left(a_m\cos\left(\omega_m(t-\tau_m)\right) + b_m\sin\left(\omega_m(t-\tau_m)\right)\right) + \mu + \left(\alpha + \beta t\right) && \left[u\right] \label{eq:racekbl:reconstruction}
\end{flalign}
In \eqref{eq:racekbl:reconstruction}, \(\omega_m\) are the three angular frequencies with the
highest periodogram power, \(a_m\) and \(b_m\) are the jointly fitted cosine and sine coefficients,
\(\tau_m\) are the phase offsets used to align each basis function, \(\mu\) is the mean of the
detrended series, and \(\alpha + \beta t\) is the linear trend that was removed before the spectral
analysis. This overlay shares the same history window as the wind speed plot, so it updates whenever
new samples arrive or the history slider changes.
