% !TeX root = master.tex

\section{VMG Hunter (RaceVMG)}

VMG Hunter works without a wind sensor. We start with the upwind case and assume the
boat is sailing in a narrow range of true wind angles (TWA), roughly \(35^\circ\) to
\(50^\circ\). The user selects an approximate TWA, and the app estimates whether a small
change in heading should be a \emph{bear away} or a \emph{head up}.

\subsection{Speed--heading trade-off}

For a given TWA \(\psi\), the upwind velocity made good is:
\[
\mathrm{VMG}(\psi) = v(\psi)\cos\psi
\]

For a small change \(d\psi\), the first-order change is:
\[
\frac{d}{d\psi}\mathrm{VMG} =
\frac{dv}{d\psi}\cos\psi - v\sin\psi
\]

At the break-even point (no VMG gain or loss), this derivative is zero:
\[
\frac{dv}{d\psi} = v\tan\psi
\]

This is the trade-off slope between speed and heading. We estimate
\(\frac{dv}{d\psi}\) from recent data and compare it to the break-even slope. If the
estimated slope is greater than the break-even slope, bearing away is faster; if it is
smaller, heading up is faster.

Because our heading samples are handled in degrees in the UI and estimator, the
comparison uses:
\[
\frac{dv}{d(\mathrm{deg})} = v\tan\psi \cdot \frac{\pi}{180}
\]

\subsection{Estimating the slope (RLS with forgetting)}

We estimate the local slope using recursive least squares (RLS) with exponential
forgetting so it behaves like an IIR filter and responds smoothly:
\begin{itemize}
\tightlist
\item use GPS speed (later: Kalman-filtered) and a heading estimate
\item unwrap heading to avoid \(360^\circ\) discontinuities
\item update the RLS state \((\theta, P)\) with a forgetting time constant
  \(\tau \approx 10\,\mathrm{s}\)
\end{itemize}

Heading source. When IMU is enabled, we integrate the gyro yaw rate to track heading
changes and use that heading in the regressor. GPS course is used to correct long-term
drift only when speed exceeds \(2\,\mathrm{kn}\) and GPS accuracy is acceptable. If IMU is
disabled, or no IMU heading is available yet, we fall back to GPS heading.

Let \(h_k\) be heading (degrees) and \(v_k\) speed (m/s) at sample \(k\). We use unwrapped
heading relative to a fixed reference so the regressor stays small:
\(h_k = \psi_k - \psi_0\). This shift does not change the slope.

We fit a line \(v_k = b_k + s_k h_k + e_k\), written in vector form:
\[
v_k = x_k^T \theta_k + e_k,\quad
x_k = \begin{bmatrix} 1 \\ h_k \end{bmatrix},\quad
\theta_k = \begin{bmatrix} b_k \\ s_k \end{bmatrix}
\]

With forgetting factor \(\lambda_k = e^{-\Delta t_k/\tau}\), the RLS recursion is:
\[
K_k = \frac{P_{k-1} x_k}{\lambda_k + x_k^T P_{k-1} x_k}
\]
\[
\theta_k = \theta_{k-1} + K_k \left(v_k - x_k^T \theta_{k-1}\right)
\]
\[
P_k = \frac{1}{\lambda_k}\left(P_{k-1} - K_k x_k^T P_{k-1}\right)
\]
where the residual is \(e_k = v_k - x_k^T \theta_{k-1}\).

The slope estimate at sample \(k\) is the second component:
\[
s_k = \theta_{k,2}
\]
when the heading excitation is large enough to be reliable.

\subsection{Confidence band and tack handling}

Confidence band. We compute a simple standard error for the slope:
\[
\sigma_{s,k} \approx \sqrt{\sigma_{e,k}^2 \, P_{k,22}}
\]
where \(\sigma_{e,k}^2\) is an exponentially weighted residual variance, for example:
\[
\sigma_{e,k}^2 = (1-\alpha_k)\sigma_{e,k-1}^2 + \alpha_k e_k^2,\quad
\alpha_k = 1 - e^{-\Delta t_k/\tau}
\]
The UI maps \(\sigma_{s,k}\) to a confidence band width. If GPS is stale or accuracy is
poor, the bar is desaturated to signal low trust.

Tack handling. The indicator is directional: on port tack the left/right meaning flips.
We implement this by negating the slope before comparing to the break-even value and
swapping the labels accordingly.
