% !TeX root = master.tex

\section{RaceLifter (headers and lifts)}

This section explains how RaceLifter turns heading samples into a stable lift/shift view.
The aim is to show how much the heading deviates from a smoothed reference, without the
plot jumping around when the window size changes or when a new sample lands.

\subsection{Heading baseline and deviation}

This subsection defines the low-pass baseline heading and the deviation that is plotted.

RaceLifter collects heading samples from the selected source (boat model or GPS) whenever
speed exceeds a small threshold (about \(0.5\,\mathrm{m/s}\)), so drifting or stationary
noise does not dominate the plot. The window slider sets the time constant used for the
baseline filter.

\begin{flalign}
\hspace{5em}\alpha_k &= \frac{\Delta t_k}{\tau+\Delta t_k} && \left[1\right] \label{eq:lifter:alpha}
\end{flalign}
In \eqref{eq:lifter:alpha}, \(\alpha_k\) is the per-sample gain, \(\Delta t_k\) is the
sample interval, the subscript \(k\) is the sample index, and \(\tau\) is the user-selected
window time constant.

\begin{flalign}
\hspace{5em}\overline{h}_k &= \overline{h}_{k-1}+\alpha_k\,\mathrm{wrap}\left(h_k-\overline{h}_{k-1}\right) && \left[\mathrm{deg}\right] \label{eq:lifter:hbar}
\end{flalign}
In \eqref{eq:lifter:hbar}, \(h_k\) is the instantaneous heading in degrees and
\(\overline{h}_k\) is the smoothed baseline heading. The wrap operator returns the
shortest signed angular difference so the baseline follows headings across
\(0^\circ\) cleanly.

\begin{flalign}
\hspace{5em}\Delta h_k &= \mathrm{wrap}\left(h_k-\overline{h}_k\right) && \left[\mathrm{deg}\right] \label{eq:lifter:dh}
\end{flalign}
In \eqref{eq:lifter:dh}, \(\Delta h_k\) is the signed deviation from the baseline in
degrees. This is the quantity drawn in the plot.

\subsection{Plot layout and colors}

This subsection describes the vertical time layout and the color convention.

\begin{flalign}
\hspace{5em}T_{\mathrm{window}} &= \tau && \left[\mathrm{s}\right] \label{eq:lifter:window}
\end{flalign}
In \eqref{eq:lifter:window}, \(T_{\mathrm{window}}\) is the time span shown in the plot. The
vertical axis runs from oldest samples at the bottom to ``now'' at the top. The horizontal
axis is the signed deviation \(\Delta h\): line segments are green when they fall to the
right of center and red when they fall to the left, with the area to the centerline filled
in the same color. The center line marks zero deviation.

\subsection{Debug playback}

This subsection outlines how the debug mode feeds the same plotting pipeline.

When debug mode is enabled, replay is controlled from the RaceTools page. The replay
button opens a list of recorded sessions pulled from files listed in
\texttt{replay/manifest.json}, and the selected session is streamed back into the app as
time-stamped samples. RaceLifter receives those samples through the same heading pipeline
as live data, so the baseline and plot behavior stay consistent. The replay speed slider
adjusts the playback time base, while the render cadence stays capped to the same pacing
used on the water, keeping the plot stable during tuning.
