% !TeX root = master.tex

\section{RaceVMG (VMG evaluator)}

This section describes how RaceVMG turns a stream of heading and speed samples into a
simple, stable evaluation: is the most recent window better or worse than the one just
before it, and by how much. The goal is not to estimate true wind, but to provide a
consistent reference so changes in trim or steering show up as a clear percentage.

\subsection{Modes, wind axis, and sign}

This subsection explains how the mode and tack selections define the reference direction
used to compute VMG.

RaceVMG does not estimate wind direction directly. Instead, it constructs a wind axis
from the older of the two evaluation windows. We take the time-weighted mean heading in
that older window and then apply a simple mode rule. In beating mode we assume an upwind
TWA and offset the mean by \(\pm\)TWA depending on tack (port is negative, starboard is
positive). In running mode we use a downwind TWA slider and apply the same tack offset.
In reaching mode we assume no TWA and use the mean heading itself as the reference axis.

The VMG at any moment is the component of speed along that axis:
\[
\mathrm{VMG} = v\cos\Delta\psi
\]
where \(\Delta\psi\) is the angular difference between heading and the chosen axis. For
running, VMG is expected to be negative because the axis points downwind. To keep the
readout intuitive (more running is better), we flip the sign when computing the change so
a more negative VMG counts as an improvement.

\subsection{Windowed comparison and percent change}

This subsection describes how two adjacent windows are compared to produce a percent
change.

We maintain a rolling history of heading and speed samples and compare two windows of
length \(W\). For beating and running, \(W\) is a fixed \(30\,\mathrm{s}\) so the feedback
feels immediate. For reaching, the window length follows half of the selected plot window
so slower changes remain visible without amplifying noise. Because GPS samples are not
uniformly spaced, we compute window averages with time-weighted linear interpolation
between samples rather than assuming fixed-rate data.

Let \(\overline{V}_\text{prev}\) and \(\overline{V}_\text{curr}\) be the average VMG values
in the older and newer windows after the running-mode sign correction. The VMG change is
reported as a percentage:
\[
\Delta_\% = \frac{\overline{V}_\text{curr} - \overline{V}_\text{prev}}{|\overline{V}_\text{prev}|}\times 100
\]
We suppress the output if the baseline is too small and clamp the displayed value to a
reasonable range so outliers do not dominate the plot.

\subsection{Plot bins and baseline}

This subsection explains how the VMG change history becomes the vertical bar plot.

The plot window is user-selected from one to five minutes and is divided into a fixed
number of time bins (30). Each computed \(\Delta_\%\) value is added to the current bin
based on its timestamp, and the bin value is the mean of all samples it contains. The
baseline for centering the plot is the mean of the oldest half of the bins, which keeps
the reference stable and avoids bias from the active bin that is still accumulating
samples. The plot is scaled symmetrically, rounded to fixed step sizes so the grid feels
stable, and the active bin is clipped to the current scale to prevent temporary spikes
from rescaling the entire view.

\subsection{Heading sources and device motion}

This subsection describes how heading is sourced and how the device motion option
interacts with GPS.

RaceVMG can use the boat model (Kalman) heading or raw GPS heading, depending on the user
selection. When the device motion sensor is enabled, we integrate yaw rate to track
short-term heading changes and blend it back toward GPS heading when speed is sufficient
and accuracy is acceptable. This improves responsiveness when the phone is rigidly
mounted to the boat, but it can degrade estimates if the phone is hand-held or moved
relative to the boat, which is why the UI warns about mounting.
